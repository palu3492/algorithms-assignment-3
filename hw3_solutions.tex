%\documentclass[11pt]{exam}
\documentclass[11pt,answers]{exam}

%Add the ``answers'' flag as above to render your solutions (everything in a
%solution environment)
%Otherwise, only the questions will be rendered

% This header contains many macros useful in Math and Computer Science.
% They may be useful for you to use, and to help learn LaTeX. 
% You don't have to learn them all, but I may use them in the questions below.

%This header file was created and compiled from many sources by Prateek Bhakta, 
%1/1/2015, with heavy influence from the header files of Chris Peikert.
\usepackage[hmargin=1in, vmargin=1in]{geometry}
\usepackage{verbatim}
\usepackage{amsmath}
\usepackage{amssymb}
\usepackage{amsfonts}
\usepackage{mathrsfs}
\usepackage{amsbsy}
\usepackage[svgnames]{xcolor}
\usepackage{textcomp}
\usepackage{mathtools}
\usepackage[amsmath,thmmarks,thref]{ntheorem}
\usepackage{xspace}
\usepackage{algorithm}
\usepackage[noend]{algpseudocode}
\usepackage{enumerate}
\usepackage{graphicx}
\usepackage[hypcap=true]{caption}
\usepackage[hypcap=true]{subcaption}
\usepackage{hyperref}
\usepackage{pdfsync}
\usepackage[expansion=false]{microtype}
\usepackage{tikz}
\usetikzlibrary{arrows}

\hypersetup{%
  colorlinks=true,% hyperlinks will be colored
}

\setlength{\textheight}{9in}
\setlength{\baselineskip}{0in}

% Generate the header of the homeworks
\newcommand{\hwheader}[5]{
  \chead{\Large \textbf{#1}}

  \lhead{\small
    \textbf{#2}\\ #5} %Course / Semester

  \rhead{\small \textbf{Instructor: #3 }\\Student: #4} %Instructor / Student

  \setlength{\headheight}{26pt}
  \setlength{\headsep}{16pt}
  
  \headrule
}

% Theorem related latex commands - requires the ntheorem package, not amsthm
% You won't be needing to do much theorem formatting in homeworks - these are
% more useful for papers. This is designed for the exam doctype, and theorems
% will renumber with each question reset.

\theoremstyle{plain}            % following are "theorem" style
\theoremheaderfont{\it\bfseries}
\theorembodyfont{\upshape}
\theoremseparator{:}
\theoremstyle{plain}            % following are "theorem" style
\newtheorem{theorem}{Theorem}[question]
\newtheorem{lemma}[theorem]{Lemma}
\newtheorem{corollary}[theorem]{Corollary}
\newtheorem{proposition}[theorem]{Proposition}
\newtheorem{claim}[theorem]{Claim}
\newtheorem{openproblem}[theorem]{Open Problem}
\newtheorem{conjecture}[theorem]{Conjecture}
\newtheorem{example}[theorem]{Example}
\newtheorem*{remark}[theorem]{Remark} %Do not advance numbers
\newtheorem*{note}[theorem]{Note} %Do not advance numbers
\newtheorem{exercise}[theorem]{Exercise}

\theoremsymbol{\ensuremath{\diamondsuit}}
\theoremheaderfont{\bfseries}
\theorembodyfont{\upshape}
\theoremseparator{.}
\theoremprework{\bigskip\hrule}
\theorempostwork{\hrule\bigskip}
\newtheorem{definition}{Definition}[question]

\theoremstyle{nonumberplain}
\theoremindent0cm
\theoremheaderfont{\scshape}
\theorembodyfont{\upshape}
\theoremseparator{:}
\theoremsymbol{\rule{1ex}{1ex}}
\theoremindent0.5cm
\newtheorem{proof}{Proof}[question]

\numberwithin{equation}{question}

%Shorthand for fitted sets, lists, etc. "left-right" pairs of symbols

% inner product
\DeclarePairedDelimiter\inner{\langle}{\rangle}
% absolute value
\DeclarePairedDelimiter\abs{\lvert}{\rvert}
% a set
\DeclarePairedDelimiter\set{\{}{\}}
% parens
\DeclarePairedDelimiter\parens{(}{)}
% tuple, alias for parens
\DeclarePairedDelimiter\tuple{(}{)}
% square brackets
\DeclarePairedDelimiter\bracks{[}{]}
% rounding off
\DeclarePairedDelimiter\round{\lfloor}{\rceil}
% floor function
\DeclarePairedDelimiter\floor{\lfloor}{\rfloor}
% ceiling function
\DeclarePairedDelimiter\ceil{\lceil}{\rceil}
% length of some vector, element
\DeclarePairedDelimiter\length{\lVert}{\rVert}
% norm (same as length)
\DeclarePairedDelimiter\norm{\lVert}{\rVert}
% "lifting" of a residue class
\DeclarePairedDelimiter\lift{\llbracket}{\rrbracket}
\DeclarePairedDelimiter\len{\lvert}{\rvert}

%multipart functions
\newcommand{\multi}[1]{
\left\{
\begin{array}{rl}
  #1
\end{array} \right.
}

%set lists
\newcommand{\setto}[2][1]{\set{#1,\ldots,#2}}
\newcommand{\listset}[3]{\set{#1_{#2},\ldots,#1_{#3}}}
\newcommand{\ltset}[3]{\set{#1_{#2},\ldots,#1_{#3}}}
\newcommand{\listsum}[3]{#1_{#2}+ \ldots + #1_{#3}}
\newcommand{\ltsum}[3]{#1_{#2}+ \ldots + #1_{#3}}
\newcommand{\listprod}[3]{#1_{#2}\cdot \ldots \cdot #1_{#3}}
\newcommand{\ltprod}[3]{#1_{#2}\cdot \ldots \cdot #1_{#3}}
\newcommand{\listand}[3]{#1_{#2}\land \ldots \land #1_{#3}}
\newcommand{\ltand}[3]{#1_{#2}\land \ldots \land #1_{#3}}
\newcommand{\listor}[3]{#1_{#2}\lor \ldots \lor #1_{#3}}
\newcommand{\ltor}[3]{#1_{#2}\lor \ldots \lor #1_{#3}}

%Shorthand for ``fancy type'' - use for sets, etc.
% blackboard symbols

%Some are letters are excluded because they often mean other things -
%\E for Expected Value, etc.
\newcommand{\bb}[1]{\ensuremath{\mathbb{#1}}}
\newcommand{\A}{\bb{A}}
\newcommand{\B}{\bb{B}}
\newcommand{\C}{\bb{C}}
\newcommand{\D}{\bb{D}}
\newcommand{\F}{\bb{F}}
\newcommand{\G}{\bb{G}}
\newcommand{\J}{\bb{J}}
\newcommand{\M}{\bb{M}}
\newcommand{\N}{\bb{N}}
%renew overwrites the old macro - \L used by latex isn't useful to me.
\renewcommand{\L}{\bb{L}}
\newcommand{\Q}{\bb{Q}}
\newcommand{\R}{\bb{R}}
\newcommand{\p}{\bb{P}}
\newcommand{\T}{\bb{T}}
\newcommand{\U}{\bb{U}}
\newcommand{\V}{\bb{V}}
\newcommand{\X}{\bb{X}}
\newcommand{\Z}{\bb{Z}}
\newcommand{\QR}{\bb{QR}}

% sets in calligraphic type - I don't use this much
\newcommand{\calD}{\ensuremath{\mathcal{D}}}
\newcommand{\calF}{\ensuremath{\mathcal{F}}}
\newcommand{\calG}{\ensuremath{\mathcal{G}}}
\newcommand{\calH}{\ensuremath{\mathcal{H}}}
\newcommand{\calX}{\ensuremath{\mathcal{X}}}
\newcommand{\calY}{\ensuremath{\mathcal{Y}}}

% sets in script
\newcommand{\scr}[1]{\ensuremath{\mathscr{#1}}}
\newcommand{\SA}{\scr{A}}
\newcommand{\SB}{\scr{B}}
\newcommand{\SC}{\scr{C}}
\newcommand{\SD}{\scr{D}}
\newcommand{\SE}{\scr{E}}
\newcommand{\SF}{\scr{F}}
\newcommand{\SG}{\scr{G}}
\newcommand{\SJ}{\scr{J}}
\newcommand{\SK}{\scr{K}}
\newcommand{\SL}{\scr{L}}
\newcommand{\SM}{\scr{M}}
\newcommand{\SN}{\scr{N}}
%Avoid using O's in math! It looks like zero!
\newcommand{\SP}{\scr{P}}
\newcommand{\SQ}{\scr{Q}}
\newcommand{\SR}{\scr{R}}
%SS is defined by Latex, I don't want to overwrite because I don't know why
\newcommand{\ST}{\scr{T}}
\newcommand{\SU}{\scr{U}}
\newcommand{\SV}{\scr{V}}
\newcommand{\SW}{\scr{W}}
\newcommand{\SX}{\scr{X}}
\newcommand{\SY}{\scr{Y}}
\newcommand{\SZ}{\scr{Z}}

%Linear Algebra
% macros for matrices and vectors (bolded)

\newcommand{\matA}{\ensuremath{\mathbf{A}}}
\newcommand{\matB}{\ensuremath{\mathbf{B}}}
\newcommand{\matC}{\ensuremath{\mathbf{C}}}
\newcommand{\matD}{\ensuremath{\mathbf{D}}}
\newcommand{\matE}{\ensuremath{\mathbf{E}}}
\newcommand{\matF}{\ensuremath{\mathbf{F}}}
\newcommand{\matG}{\ensuremath{\mathbf{G}}}
\newcommand{\matH}{\ensuremath{\mathbf{H}}}
\newcommand{\matI}{\ensuremath{\mathbf{I}}}
\newcommand{\matJ}{\ensuremath{\mathbf{J}}}
\newcommand{\matK}{\ensuremath{\mathbf{K}}}
\newcommand{\matL}{\ensuremath{\mathbf{L}}}
\newcommand{\matM}{\ensuremath{\mathbf{M}}}
\newcommand{\matN}{\ensuremath{\mathbf{N}}}
\newcommand{\matP}{\ensuremath{\mathbf{P}}}
\newcommand{\matQ}{\ensuremath{\mathbf{Q}}}
\newcommand{\matR}{\ensuremath{\mathbf{R}}}
\newcommand{\matS}{\ensuremath{\mathbf{S}}}
\newcommand{\matT}{\ensuremath{\mathbf{T}}}
\newcommand{\matU}{\ensuremath{\mathbf{U}}}
\newcommand{\matV}{\ensuremath{\mathbf{V}}}
\newcommand{\matW}{\ensuremath{\mathbf{W}}}
\newcommand{\matX}{\ensuremath{\mathbf{X}}}
\newcommand{\matY}{\ensuremath{\mathbf{Y}}}
\newcommand{\matZ}{\ensuremath{\mathbf{Z}}}
\newcommand{\matzero}{\ensuremath{\mathbf{0}}}

\newcommand{\veca}{\ensuremath{\mathbf{a}}}
\newcommand{\vecb}{\ensuremath{\mathbf{b}}}
\newcommand{\vecc}{\ensuremath{\mathbf{c}}}
\newcommand{\vecd}{\ensuremath{\mathbf{d}}}
\newcommand{\vece}{\ensuremath{\mathbf{e}}}
\newcommand{\vecf}{\ensuremath{\mathbf{f}}}
\newcommand{\vecg}{\ensuremath{\mathbf{g}}}
\newcommand{\vech}{\ensuremath{\mathbf{h}}}
\newcommand{\veck}{\ensuremath{\mathbf{k}}}
\newcommand{\vecm}{\ensuremath{\mathbf{m}}}
\newcommand{\vecp}{\ensuremath{\mathbf{p}}}
\newcommand{\vecq}{\ensuremath{\mathbf{q}}}
\newcommand{\vecr}{\ensuremath{\mathbf{r}}}
\newcommand{\vecs}{\ensuremath{\mathbf{s}}}
\newcommand{\vect}{\ensuremath{\mathbf{t}}}
\newcommand{\vecu}{\ensuremath{\mathbf{u}}}
\newcommand{\vecv}{\ensuremath{\mathbf{v}}}
\newcommand{\vecw}{\ensuremath{\mathbf{w}}}
\newcommand{\vecx}{\ensuremath{\mathbf{x}}}
\newcommand{\vecy}{\ensuremath{\mathbf{y}}}
\newcommand{\vecz}{\ensuremath{\mathbf{z}}}
\newcommand{\veczero}{\ensuremath{\mathbf{0}}}


%Shorthands
\newcommand{\bit}{\ensuremath{ {\set{0,1}}} }
\newcommand{\booln}{\bit^n}
\newcommand{\zo}{\ensuremath{{ {\bracks{0,1}}} }}
\newcommand{\pmone}{\ensuremath{{ {\set{-1,1}}}}}

\makeatletter
\DeclareRobustCommand\onedot{\futurelet\@let@token\@onedot}
\def\@onedot{\ifx\@let@token.\else.\null\fi\xspace}
\def\eg{{e.g}\onedot} \def\Eg{{E.g}\onedot}
\def\ie{{i.e}\onedot} \def\Ie{{I.e}\onedot}
\def\cf{{c.f}\onedot} \def\Cf{{C.f}\onedot}
\def\st{{s.t}\onedot}
\def\io{{i.o}\onedot}
\def\as{{a.s}\onedot}
\def\etc{{etc}\onedot}
\def\vs{{vs}\onedot}
\def\wrt{{w.r.t}\onedot}
\def\dof{{d.o.f}\onedot}
\def\etal{{et al}\onedot}
\def\th{\ensuremath{^{th}}}
\makeatother
\newcommand{\bs}{\backslash}
%\newcommand{\qed}{\hfill $\square$ }

\def\iid{i.i.d\onedot}
\def\pdf{p.d.f\onedot}
\def\cdf{c.d.f\onedot}
\def\mgf{m.g.f\onedot}

\newcommand{\orr}{\vee}
\newcommand{\andd}{\wedge}


\newcommand{\inv}[1]{\frac{1}{#1}}
\newcommand{\oon}{\inv{n}}
\renewcommand{\half}{\inv{2}}
\newcommand{\third}{\inv{3}}
\newcommand{\xor}{\oplus}

% probability/distribution stuff
\DeclareMathOperator*{\E}{E}
\DeclareMathOperator*{\Var}{Var}
\DeclareMathOperator*{\argmin}{argmin}
\DeclareMathOperator*{\argmax}{argmax}
\DeclareMathOperator*{\Cov}{Cov}
\DeclareMathOperator*{\Corr}{Corr}

\DeclareMathOperator*{\Unif}{Unif}
\DeclareMathOperator*{\Bin}{Bin}
\DeclareMathOperator*{\Geom}{Geom}
\DeclareMathOperator*{\Pois}{Pois}
\DeclareMathOperator*{\Expo}{Exp}
\DeclareMathOperator*{\Erlang}{Erlang}
\DeclareMathOperator*{\Gam}{Gamma}
\DeclareMathOperator*{\Nor}{Norm}

\newcommand{\PR}[2][]{\P_{#1}\bracks{#2}}
\newcommand{\PRT}[2][]{\P_{#1}\bracks{\text{#2}}}
\newcommand{\PRS}[2][]{\P_{#1}\parens{#2}}
%My personal choice - use \EV and \VAR for automatic brackets, use \E and \Var
%for rare multi-line things
\newcommand{\EV}[1]{\E\bracks{#1}}
\newcommand{\VAR}[1]{\Var\bracks{#1}}
\newcommand{\COV}[1]{\Cov\bracks{#1}}
\newcommand{\CORR}[1]{\Corr\parens{#1}}

%convergence in difference ways
\newcommand{\overto}[1]{\overset{#1}{\to}}
\newcommand{\pto}{\overto{\P}}
\newcommand{\dto}{\overto{\D}}
\newcommand{\asto}{\overto{\as}}

%Graph Theory Stuff
\newcommand{\gnp}{G_{n,p}}
\newcommand{\gnnp}{G_{n,n,p}}

%Geometry
\newcommand{\oline}[1]{\overleftrightarrow{#1}}

\newcommand{\Zt}{\ensuremath{\Z_t}}
\newcommand{\Zp}{\ensuremath{\Z_p}}
\newcommand{\Zq}{\ensuremath{\Z_q}}
\newcommand{\ZN}{\ensuremath{\Z_N}}
\newcommand{\Zps}{\ensuremath{\Z_p^*}}
\newcommand{\ZNs}{\ensuremath{\Z_N^*}}
\newcommand{\JN}{\ensuremath{\J_N}}
\newcommand{\QRN}{\ensuremath{\mathbb{QR}_N}}
\newcommand{\QRp}{\ensuremath{\QR_{p}}}

%Shorthand for common CS Theory things

% asymptotic stuff
\DeclareMathOperator{\poly}{poly}
\DeclareMathOperator{\polylog}{polylog}
\DeclareMathOperator{\negl}{negl}
\newcommand{\Otil}{\ensuremath{\tilde{O}}}

% font for general-purpose algorithms
\newcommand{\algo}[1]{\ensuremath{\mathsf{#1}}}
% font for general-purpose computational problems
\newcommand{\prob}[1]{\ensuremath{\mathsf{#1}}}
% font for complexity classes
\newcommand{\class}[1]{\ensuremath{\mathsf{#1}}}

% complexity classes and languages
\renewcommand{\P}{\class{P}}
\newcommand{\NP}{\class{NP}}
\newcommand{\PH}{\class{PH}}
\renewcommand{\L}{\class{L}}
\newcommand{\NL}{\class{NL}}
\newcommand{\EXP}{\class{EXP}}
\newcommand{\coNP}{\class{coNP}}
\newcommand{\BPP}{\class{BPP}}
\newcommand{\ZPP}{\class{ZPP}}
\newcommand{\RP}{\class{RP}}
\newcommand{\coRP}{\class{coRP}}
\newcommand{\PSP}{\class{PSPACE}}
\newcommand{\NPSP}{\class{NPSPACE}}
\newcommand{\AM}{\class{AM}}
\newcommand{\MAM}{\class{MAM}}
\newcommand{\coAM}{\class{coAM}}
\newcommand{\IP}{\class{IP}}
\newcommand{\DSP}{\class{DSPACE}}
\newcommand{\DTIME}{\class{DTIME}}

%Problems and Algorithms
\newcommand{\SAT}{\prob{SAT}}
\newcommand{\coSAT}{\prob{coSAT}}
\newcommand{\SATBP}{\prob{SAT - BP}}

%Pseudocode Macros
\newcommand{\IF}[2]{ \If{#1} #2\EndIf }
\newcommand{\FOR}[2]{ \For{#1} #2 \EndFor }
\newcommand{\FORALL}[2]{ \ForAll{#1} #2 \EndFor}
\newcommand{\WHILE}[2]{ \While{#1}#2\EndWhile}
\newcommand{\REPEAT}[2]{ \Repeat#1\Until{#2}}
\newcommand{\FUNCTION}[3]{ \Function{#1}{#2}#3\EndFunction}
\algnewcommand\OR{\ensuremath{\mathbf{or}}}
\algnewcommand\AND{\ensuremath{\mathbf{and}}}
\algnewcommand\NOT{\ensuremath{\mathbf{not}}}


% Below are some additional, homework-specific Macros.

\begin{document}
%Your information goes here
\hwheader{Homework 3}{CISC 380}{Dr. Sarah Miracle}{Emma Sinn and Alex Palumbo}{}

\noindent {\bf Purpose:} 
 \begin{itemize}
\item Practice writing recurrences (i.e., recursive solutions).
 \item Learn to design and implement algorithms using dynamic programming techniques.
\item Continue to develop your algorithm analysis skills.
 \end{itemize}

\hrule
\vspace{2em}

\noindent {\bf General Homework Policies:}
\begin{itemize}
\item This homework assignment is due by the \textcolor{blue}{start of class on Monday, March 16}. In order to submit your homework, you must do two things.
\begin{enumerate}
\item Submit a soft-copy through Canvas.  For Problem 1 this should be a  \texttt{pdf}  (it should generated by modifying the the LaTeX file for this assignment).  For Problem 2 this should be the completed \texttt{Interleaver.java} file.  If you choose to complete the extra credit you will also submit the file \texttt{MaxSum.java}. Before submitting please zip all the files together and submit one \texttt{.zip} file.  See the \emph{Additional Programming Instructions} section below for more detailed instructions.
\item Bring a paper print-out of your pdf solution to class.
\end{enumerate}
\emph{Late assignments will not be accepted and will receive a 0!}
\item \textcolor{blue}{You will be assigned a partner for this assignment.}  Only one assignment should be submitted and you and your partner will both receive the same grade.   Make sure to include your partner's name on the homework.  Your assignment should be a true joint effort, equally created, and understood by both partners. 
\item You are not allowed to consult outside sources, other than the textbook, your notes, the Java API and the references linked from Canvas (i.e., no looking for answers on the internet).$^1$
\item Getting \emph{ANY} solutions from the web, previous students etc. is \emph{NOT} allowed.$^1$ 
\item You are not allowed to discuss this assignment with anyone except for your partner (if you have one) or the instructor.$^1$
\item Copying code from anywhere or anyone is not allowed (even previous code you have written).  Allowing someone to copy your code is also considered cheating.$^1$
\item Your work will be graded on correctness and clarity.  Write complete and precise answers and show all of your work.  
\item Questions marked (PRACTICE) will not be graded and do not need to be submitted. However it is highly recommended that you complete them.
\end{itemize}
\emph{$^1$See the section of the syllabus on academic dishonesty for more details.}\\
\hrule
\vspace{2em}
\noindent {\bf Homework Problems:}\\ 
\begin{questions}

\question[5]
  Recall that a subarray is similar to a subsequence except that the elements must be consecutive.  For instance, if $A$ is $[5,15,-30,10,-5,40,10]$
  then $[15,-30]$ is a subarray but $[5,15,40]$ is not.  Use {\bf dynamic programming} to give a
  linear-time algorithm for maximum sum subarray problem (seen before in class and on assignment 2):\\\\ 
  \emph{Input:} A list of numbers, $a_1, a_2, \ldots, a_n.$\\ 
  \emph{Output:} The \textbf{sum} of the subarray with maximum sum \\ 
  Note that if the array contains only negative numbers then the answer should be negative.  For example, if the input is \texttt{[-1, -2, -3]} then your method should return -1 (i.e. a subarray has length at least 1).\\

  Your solution to must use \textbf{dynamic programming} (There should be a table!) and you should write your solution in the following form:
\begin{parts}
  \part[1] Define your subproblems (the entries of the table) in words.  For
    example, in the domino tiling problem the definition is $D(i)$ = the number of ways to tile a $2\times i$ checkerboard.
     \begin{solution}
    The definition of each subproblem is \\
    \[
    S(i) = \mbox{max sum subarray ending at index } i \\
    \]
    Using Kadane's algorithm, the subarray ending at index i is either the max sum subarray ending at i-1 extended to i or it is the new max sum subarray starting and ending at index i.
  \end{solution}
  \part[2]State the recurrence for the entries of your table (NOT for the running time), and explain in
    words why it's true.  For
    example, in the domino tiling problem the recurrence for the entries is $D(i) = D(i-1) + D(i-2).$

     \begin{solution}
    The recurrence for entries of the table is \\
    \[
    S(i) = \begin{cases} 
      A[i] & i = 0 \\
      max( S(i-1) + A[i], A[i] ) & i > 0
   \end{cases}
   \]
   The first entry in the table will be the first value in the input array. The following table entries will be the maximum value
   between the value at the current index of the input array and the sum of the previous table entry added with the value at the
   current index of the input array.
  \end{solution}
  \part[1] Give pseudocode for filling in the table AND returning the solution.  Do NOT use memoization.
    \begin{solution}
        \begin{algorithmic}
        \State \underline{Input}: array $A = [a1, . . . , an]$
        \State \underline{Output}: The sum of the maximum sum subarray. \\
        \Function{MaxSubArray}{$A$}
            \State $maxSum = A[0]$ \Comment{Default max sum to the first value in input array}
            \State $table[0] = A[0]$ \Comment{The table's first value is the input array's first value}
            \State \Comment{Find the max sum subarray at each array index}
            \For{$(i = 1; i < n; i++)$}
                \State \Comment{Extend previous subarray or start new subarray}
                \State \Comment{Which ever one is greater will be new table entry}
                \State $table[i] = max(A[i], A[i] + table[i - 1]$
                \State \Comment{Max sum reevaluated based on new table entry}
                \State $maxSum = max(maxSum, table[i])$
            \EndFor
            \State \Return $maxSum$
        \EndFunction
        \end{algorithmic}
  \end{solution}
  \part[1] Analyze the running time (give a $\Theta$-bound AND justification). 
    \begin{solution}
    The $\Theta$-bound on the algorithm's running time is $\Theta(n)$. The loop iterates over each item in the input array once, doing $\Theta(n)$ work.
  \end{solution} 
	\part[1]{\bf(EXTRA CREDIT)} Implement your pseudo-code from part (c). Add code to the \texttt{maxSumSubarray} method in the \texttt{MaxSum.java} file provided with the assignment.  Note that you will receive NO credit if your code does not implement your pseudo-code and you must use dynamic programming (there must be a table!). 
\end{parts}   

    
\question[5] 
Recall the interleaving problem from class.  For strings $X=x_1 \dots x_n$, $Y=y_1 \ldots y_m$ and $Z = z_1 \ldots Z_{m+n},$ we say that Z is an \emph{interleaving} of $X$ and $Y$ if it can be obtained by interleaving the bits in $X$ and $Y$ in a way that maintains the left-to-right order of the bits in $X$ and $Y.$  For example, if $X = abc$ and $Y=dcab$ then $x_1x_2y_1x_3y_2y_3y_4=abdccab$ is an interleaving of $X$ and $Y$ whereas $acdcbab$ is not.  

We will use the following subproblem definition.  Let $I(i,j)= TRUE$ if $z_1 \ldots z_{i+j}$ is an interleaving of $x_1 \ldots x_i$ and $y_1 \ldots y_j$ and $FALSE$ otherwise.  We came up with a recurrence for $I(i,j)$ for each of the following cases.\\


\begin{enumerate}
\item $I(i,j) = I(i-1,j)$ if $z_{i+j} = x_i \neq y_j$\\ 
\item $I(i,j) =  I(i,j-1)$ if $ z_{i+j} = y_j \neq x_i$\\ 
\item $I(i,j) = I(i,j-1) \lor I(i-1,j)$  if $z_{i+j} = x_i =y_j$\\ 
\item $I(i,j) = $ False  if $z_{i+j} \neq x_i, z_{i+j} \neq y_i$\\
\end{enumerate}

You will implement a solution to the interleaving problem using dynamic programming and the above subproblem definition. For this problem you will add code to the \texttt{Interleaver.java} file provided with the assignment.  Note that you are welcome to add additional private methods or data fields but you may not modify the method signature of the constructor or the \texttt{isInterleaved} method.  Your solution should not print anything.
\begin{parts}
\part Implement the code for the above dynamic programming solution by adding code to the \texttt{isInterleaved} method.  You must use \textcolor{blue}{memoization}.
\part As a comment at the top of the \texttt{isInterleaved} method, analyze the running time of your algorithm in terms of $n = |X|$ and $m = |Y|$  (use $O$ notation and explain your answer).  Note that your implementation should have the best running time possible using the given subproblem and recurrence!
\part [1]{\bf(EXTRA CREDIT)} Write a method that returns the actual interleaving by adding code to the \texttt{getSolution} method.  Note that this method must call the method you wrote for part (a) to ``fill the table".  This method must not change the overall running time of the algorithm given above (i.e., it shouldn't take longer in terms of $O$ notation than the time to ``fill the table").  Read the documentation for the method carefully to determine what you should be returning!
\end{parts}
Hints: The solution above does not contain a base-case.  Think about what the base case should be - you will need multiple base-cases.  Also, your book (and the subproblem definition above) use strings/arrays that start at index 1.  However in Java, strings/arrays start at index 0.  When implementing your code you will need to account for this difference.\\

  
\question {\bf(PRACTICE)}
  You are going on a long trip.  You start on the road at mile post 0.  Along
  the way there are $n$ hotels, at mile posts $a_1 < a_2 < \ldots < a_n,$
  where each $a_i$ is measured from the starting point.  The only places you
  are allowed to stop are at these hotels, but you can choose which of the
  hotels you stop at.  You must stop at the final hotel (at distance $a_n$),
  which is your destination.\\

  You'd ideally like to travel 200 miles a day, but this may be impossible to
  do exactly (depending on the spacing of the hotels).  If you travel $x$
  miles during a day, the penalty for that day is $(200-x)^2.$  You want to
  plan your trip so as to minimize the total penalty - that is, the sum, over
  all travel days, of the daily penalties.  Give an efficient algorithm that
  determines the optimal sequence of hotels at which to stop.\\

\end{questions}
\noindent {\bf Additional Programming Instructions:}\\ 

\noindent Note that your code will be automatically run on a standard set of test cases.  In order to ensure that you do not lose points, follow the instructions below.
\begin{itemize}
\item Your code must compile without any errors using the version of Java on the lab computers.  If your code does not compile you will not receive any points for the assignment.
\item Do not modify any of the methods signatures (i.e. name, return type or input type).  Note that you are always welcome (and encouraged) to add additional methods but these will not be run directly by the test code.
\item You are not allowed to use packages (e.g. no statement \texttt{package ...} at the top of your file).
\item No extra folders or files in your submission.  Zip up only the files you need to submit not the folder they are in.
\item Your solution should not print anything unless explicitly instructed to.
\end{itemize}

\hrule
\vspace{2em}
\noindent {\bf Grading Criteria:}\\

\vspace{.5em}

\noindent Your work will be graded on both correctness {\bf and clarity}.  Write complete and precise answers and show all of your work.  Your pseudo-code and proofs should follow the guidelines posted on Canvas and discussed in class.\\

 \noindent {{\bf Grading Criteria for Problem 1:}}\\
   
\begin{center}
  \begin{tabular}{| c | p{10cm} |}
	\hline
	Component & Requirement for Full Credit\\
    \hline
    Subproblem Definition  (1pt) & The definition is clear.  It includes full sentences and should be easily understood by any student in the class. \\ \hline
  Recurrence  (2pts) & The recurrence is clear, concise and correct.  Make sure to include at least one base case!  It is described in full sentences and should be easily understood by any student in the class. It includes a justification for WHY the recurrence is correct. \\ \hline
	Pseudo-code  (1pt) & The pseudo-code is clearly written following the guidelines discussed in class.\\ \hline
	Running Time Analysis  (1pt) & The solution gives both a bound and an explanation.  Both are clearly written and correct (for the algorithm given!).  \\ \hline
  Implementation (EXTRA CREDIT) (1pt) & Your code will be run on a standard set of test cases.  Make sure to test your code thoroughly!  Note that you will lose points if you do not follow the general style guidelines given in the syllabus or if your implementation changes the running time of your pseudo-code.  You will receive NO credit if your code does not implement your pseudo-code and you must use dynamic programming (there must be a table!).  \\ \hline
		
  \end{tabular}
\end{center}
   \noindent {{\bf Grading Criteria for Problem 2:}}\\
   
\begin{center}
  \begin{tabular}{| c | p{9cm} |}
	\hline
	Component & Description 	\\
    \hline
    Style \& Documentation  (1 pts) & I'll be watching for style issues as well as correct output.  See the syllabus for some general style guidelines (e.g. your code should be well-documented). 
		\\ \hline
		\textcolor{black}{Big-O Analysis (1 pt)} & \textcolor{black}{Make sure to justify your analysis as well as give a bound (i.e., explain where your bound comes from and why it is correct).  Your implementation must have the best running time possible using the given subproblem and recurrence.}\\
		\hline
 Correct Output on Test Cases* (3 pt) & Your code will be run on a standard set of test cases.  Make sure to test your code thoroughly! \\ \hline
	EXTRA CREDIT (1pt) & Your \texttt{getSolution} method will be run on a standard set of test cases.  Make sure to test your code thoroughly!  Note that you will lose points if you do not follow the general style guidelines given in the syllabus or if your implementation changes the running time of the algorithm. Your method must use the table created by your \texttt{isInterleaved} method to find the solution. \\ \hline
  \end{tabular}
\end{center}
*Note that if your code does not implement the given dynamic programming solution you will receive no credit regardless of whether the test cases execute correctly.  Remember you are required to use \textcolor{blue}{memoization}!

\end{document}
